\documentclass[11pt,a4paper]{scrartcl}

% - Base packages -
\usepackage[utf8]{inputenc}
\usepackage[T1]{fontenc}
\usepackage[english]{babel}
\usepackage[a4paper, top=2.5cm, bottom=2.5cm, left=2.8cm, right=2.8cm, headheight=36.5pt, footskip=1.5cm]{geometry}
\usepackage{amsmath,amssymb}
\usepackage{graphicx}
\usepackage{enumitem}
\usepackage{hyperref}
\usepackage[automark,headsepline]{scrlayer-scrpage}
\usepackage{xcolor}
\usepackage{tcolorbox}
\usepackage{listings}
\usepackage{microtype}
\usepackage{tikz}  % For enhanced graphics and borders
\usetikzlibrary{shadows,patterns}

% Professional monospace font setup
\usepackage{inconsolata}  % Clean, crisp monospace font
\usepackage[scaled=1.0]{helvet}  % For sans-serif headers
\renewcommand{\familydefault}{\ttdefault}  % Use monospace as main font
\usepackage[libertine]{newtxmath}  % Math font that pairs well with monospace

% Professional grayscale color scheme
\definecolor{titlecolor}{RGB}{20, 20, 20}
\definecolor{sectioncolor}{RGB}{40, 40, 40}
\definecolor{accentcolor}{RGB}{60, 60, 60}
\definecolor{rulecolor}{RGB}{100, 100, 100}
\definecolor{lightgray}{RGB}{245, 245, 245}
\definecolor{codebg}{RGB}{250, 250, 250}
\definecolor{codehighlight}{RGB}{240, 240, 245}
\definecolor{shadowcolor}{RGB}{220, 220, 220}

% Custom command for inline code paths with subtle highlight
\newcommand{\codepath}[1]{\colorbox{codehighlight}{\texttt{#1}}}

% Section divider command
\newcommand{\sectiondivider}{%
  \vspace{0.2em}%
  \noindent\textcolor{rulecolor}{\rule{0.3\textwidth}{0.4pt}}%
  \vspace{0.3em}%
}

% Hyperref setup - understated
\hypersetup{
    colorlinks=true,
    linkcolor=black,
    urlcolor=accentcolor,
    citecolor=black,
    pdfborder={0 0 0}
}

% Code listing setup - minimal and crisp
\lstset{
    language=Python,
    basicstyle=\ttfamily\small,
    breaklines=true,
    frame=none,
    backgroundcolor=\color{codebg},
    xleftmargin=1em,
    framexleftmargin=0em,
    columns=fullflexible,
    showstringspaces=false,
    keywordstyle=\bfseries,
    commentstyle=\color{accentcolor}\itshape,
    stringstyle=\itshape,
    numbers=left,
    numberstyle=\tiny\color{rulecolor},
    numbersep=1em
}
\lstdefinelanguage{Dockerfile}{
  morekeywords={FROM,WORKDIR,COPY,RUN,CMD},
  sensitive=true,
  morecomment=[l]{\#},
  morestring=[b]",
}

% Professional tcolorbox styles with shadows
\tcbuselibrary{listings,breakable,skins}

\tcbset{
    modern/.style={
        enhanced,
        sharp corners,
        colback=codebg,
        colframe=rulecolor,
        boxrule=0.5pt,
        arc=0mm,
        top=8pt,
        bottom=8pt,
        left=12pt,
        right=12pt,
        breakable,
        before skip=10pt,
        after skip=10pt,
        drop fuzzy shadow=shadowcolor
    }
}

% Section formatting - minimal and professional with accent bars
\setkomafont{section}{\ttfamily\LARGE\bfseries\color{sectioncolor}}
\setkomafont{subsection}{\ttfamily\Large\bfseries\color{sectioncolor}}
\setkomafont{subsubsection}{\ttfamily\large\bfseries\color{accentcolor}}

% Add subtle bars before sections
\RedeclareSectionCommand[
  beforeskip=1.0\baselineskip,
  afterskip=.5\baselineskip
]{section}

\RedeclareSectionCommand[
  beforeskip=0.8\baselineskip,
  afterskip=.3\baselineskip
]{subsection}

% List spacing with improved bullets
\setlist{nosep, leftmargin=1.8em}
\setlist[itemize,1]{label={\textcolor{rulecolor}{\small\textbullet}}}
\setlist[enumerate,1]{label={\textcolor{sectioncolor}{\arabic*.}}}

% Paragraph spacing - compact layout
\setlength{\parskip}{0.4em}
\setlength{\parindent}{0pt}

\setlength{\footheight}{25pt}

\clearpairofpagestyles

% Enhanced header with subtle line
\chead{%
  \raisebox{-2pt}{\includegraphics[height=1.1cm]{logo.png}}
}
\cfoot{%
  \vspace{-0.5em}%
  \textcolor{rulecolor}{\rule{4cm}{0.3pt}}\\[0.3em]%
  \ttfamily\small\pagemark%
}
\renewcommand*{\titlepagestyle}{scrheadings}

\addtokomafont{title}{\vspace*{-2em}}
\addtokomafont{author}{\vspace*{-0.5em}}
\addtokomafont{date}{\vspace*{-0.5em}}

% - Title and authors -
\title{\ttfamily\Huge \textbf{Scotland Yard}\\[0.3em]
\LARGE Multi-Agent Mechanism Design}

\author{
\ttfamily Collective Intelligence - Spring 2026
}
\date{\ttfamily Teams of 2-3 students | 6-8 weeks}

\begin{document}
\maketitle

\vspace{-3em}

\begin{center}
    \includegraphics[width=0.4\textwidth]{scotland_yard.png}
\end{center}

\section*{Overview}

This assignment extends the Scotland Yard mechanism design project with \textbf{three independent research directions}. The codebase provides a working multi-agent RL framework with GNN/MAPPO agents, graph environments, and training infrastructure. Your team will choose \textbf{one task} to investigate deeply, implementing new algorithms, architectures, mechanisms and producing rigorous experimental results.

\begin{tcolorbox}[modern, title=\ttfamily\textbf{What is already provided}]
\begin{itemize}[leftmargin=1.2em, itemsep=4pt]
    \item \textbf{Environment}: Scotland Yard on procedural graphs (PettingZoo + TorchRL)
    \item \textbf{Agents}: GNN and MAPPO implementations with action masking
    \item \textbf{Training}: Self-play loops, curriculum learning, reward shaping
    \item \textbf{Infrastructure}: Hydra configs, Docker, visualization, unit tests
    \item \textbf{Evaluation}: Metrics system, exploitability tests, OOD evaluation
\end{itemize}
\end{tcolorbox}

\begin{tcolorbox}[modern, title=\ttfamily\textbf{Teams \& Timeline}]
\begin{itemize}[leftmargin=1.2em, itemsep=4pt]
    \item \textbf{3 teams} of 2-3 students each
    \item \textbf{Deadline}: 6-8 weeks from start
    \item \textbf{Each team chooses ONE task} from the three options below
\end{itemize}
\end{tcolorbox}

\section*{Task Options}
\addcontentsline{toc}{section}{Task Options}

\subsection*{Task 1: Population-Based Self-Play \& Robustness}
\textbf{Research Question}: \textit{Are mechanisms trained against diverse opponent populations more robust and less exploitable than single-opponent self-play?}

\subsubsection*{Hypothesis}
Maintaining a population of diverse agents (varying strategies, skill levels) during training produces mechanisms that generalize better to unseen opponents and are harder to exploit.

\subsubsection*{Implementation Requirements}
\begin{enumerate}[leftmargin=1.4em]
    \item \textbf{Population Manager} (\codepath{src/training/population\_trainer.py}):
    \begin{itemize}
        \item Maintain pool of 5-10 checkpoint policies for both MrX and Police
        \item Implement matchmaking: round-robin, skill-based pairing, or Elo-based
        \item Periodic checkpoint saving with diversity metrics
    \end{itemize}
    
    \item \textbf{Exploitability Testing} (\codepath{src/eval/exploitability.py} extension):
    \begin{itemize}
        \item Best-response training: freeze target mechanism, optimize attacker
        \item Measure win-rate shift under exploitation attempts
        \item Compare single-agent vs population-trained exploitability
    \end{itemize}
    
    \item \textbf{Diversity Metrics}:
    \begin{itemize}
        \item Behavioral diversity: action entropy, trajectory clustering
        \item Strategic diversity: response to different opponent styles
        \item Performance spread: Elo ratings, win-rate distributions
    \end{itemize}
\end{enumerate}

\subsubsection*{Technical Guidance}
\begin{itemize}[leftmargin=1.2em]
    \item Use \texttt{PopulationTrainer} class inheriting from \texttt{BaseTrainer}
    \item Store population in \codepath{src/artifacts/populations/}
    \item Config: \codepath{src/configs/training/population.yaml} with pool sizes, matchmaking strategy
    \item Plot: exploitability curves, diversity metrics over training, population skill distribution
\end{itemize}

\subsubsection*{Expected Deliverables}
\begin{itemize}[leftmargin=1.2em]
    \item Working population trainer with 3+ matchmaking strategies
    \item Exploitability comparison: single-agent SP vs population SP (5+ seeds)
    \item 4-6 plots: exploitability over training, diversity metrics, Elo distributions, behavioral clustering
    \item 2-3 GIFs: population agents playing, exploitation attempts
    \item Analysis: when does population training help? Failure cases?
\end{itemize}

\sectiondivider

\subsection*{Task 2: Attention Mechanisms \& Architecture Ablations}
\textbf{Research Question}: \textit{Do graph attention networks learn better strategic coordination than standard message-passing GNNs?}

\subsubsection*{Hypothesis}
Attention mechanisms allow agents to selectively focus on strategically relevant teammates/opponents, improving sample efficiency and final performance compared to uniform message aggregation.

\subsubsection*{Implementation Requirements}
\begin{enumerate}[leftmargin=1.4em]
    \item \textbf{New Agent Architectures}:
    \begin{itemize}
        \item \codepath{src/agent/gat\_agent.py}: Graph Attention Networks (PyTorch Geometric)
        \item \codepath{src/agent/transformer\_agent.py}: Transformer with positional encoding on graphs
        \item Follow same interface as \texttt{GNNAgent} (inherit from \texttt{BaseAgent})
    \end{itemize}
    
    \item \textbf{Ablation Study} (\codepath{src/eval/architecture\_ablations.py}):
    \begin{itemize}
        \item Compare: GCN (baseline) vs GAT vs Transformer
        \item Vary: number of layers (2-5), hidden dims (64-256), attention heads (1-8)
        \item Fixed environment/training config for fair comparison
    \end{itemize}
    
    \item \textbf{Attention Visualization}:
    \begin{itemize}
        \item Extract attention weights at inference time
        \item Visualize which agents/nodes receive high attention
        \item Correlate attention patterns with strategic events (captures, escapes)
    \end{itemize}
\end{enumerate}

\subsubsection*{Technical Guidance}
\begin{itemize}[leftmargin=1.2em]
    \item Use \texttt{torch\_geometric.nn.GATConv} and \texttt{TransformerConv}
    \item Configs: \codepath{src/configs/agent/gat.yaml}, \codepath{transformer.yaml}
    \item Attention extraction: hook into \texttt{.forward()} or use \texttt{return\_attention\_weights=True}
    \item Compare on same compute budget (match parameter count or training time)
\end{itemize}

\subsubsection*{Expected Deliverables}
\begin{itemize}[leftmargin=1.2em]
    \item 2 new agent implementations (GAT + Transformer) with unit tests
    \item Ablation results: 3+ architectures $\times$ 3+ hyperparameter settings (5 seeds each)
    \item 4-6 plots: learning curves, sample efficiency, performance vs parameters, attention heatmaps
    \item 2-3 GIFs: attention-weighted graphs during gameplay
    \item Analysis: when does attention help? Diminishing returns? Interpretability insights?
\end{itemize}

\sectiondivider

\subsection*{Task 3: Multi-Objective Mechanisms (Fairness vs Efficiency)}
\textbf{Research Question}: \textit{Can priority edges (metro/ship) enable Pareto-optimal trade-offs between catch efficiency and detective workload fairness?}

\subsubsection*{Hypothesis}
Adding high-speed "metro" edges that MrX can use creates interesting mechanism design trade-offs: faster games but potential unfairness in which detectives get catches. Optimizing both objectives reveals a Pareto frontier.

\subsubsection*{Implementation Requirements}
\begin{enumerate}[leftmargin=1.4em]
    \item \textbf{Priority Edge System} (\codepath{src/environment/graph\_generator.py} extension):
    \begin{itemize}
        \item Generate graphs with 10-20\% "metro" edges (weight 0.5-1.0 of normal)
        \item MrX can use all edges; Police restricted to normal edges (configurable)
        \item Visualize priority edges distinctly (different colors/line styles)
    \end{itemize}
    
    \item \textbf{Multi-Objective Rewards} (\codepath{src/environment/multi\_objective\_reward.py}):
    \begin{itemize}
        \item Objective 1 (Efficiency): time-to-catch, total travel distance
        \item Objective 2 (Fairness): Gini coefficient of catches per detective, workload balance
        \item Scalarization: weighted sum $\alpha \cdot \text{eff} + (1-\alpha) \cdot \text{fair}$
    \end{itemize}
    
    \item \textbf{Pareto Frontier Exploration}:
    \begin{itemize}
        \item Train mechanisms with $\alpha \in \{0.0, 0.25, 0.5, 0.75, 1.0\}$
        \item Compare standard graphs vs metro-enhanced graphs
        \item Identify Pareto-optimal configurations
    \end{itemize}
\end{enumerate}

\subsubsection*{Technical Guidance}
\begin{itemize}[leftmargin=1.2em]
    \item Extend \texttt{GraphGenerator.generate()} with \texttt{priority\_edge\_ratio} parameter
    \item Config: \codepath{src/configs/environment/priority\_edges.yaml}
    \item Fairness metrics: Gini coefficient $G = \frac{\sum_{i,j} |c_i - c_j|}{2n\sum_i c_i}$ where $c_i$ = catches by detective $i$
    \item Use existing \texttt{RewardCalculator} as base, add multi-objective wrapper
\end{itemize}

\subsubsection*{Expected Deliverables}
\begin{itemize}[leftmargin=1.2em]
    \item Priority edge generation + visualization (clearly distinguishable in GIFs)
    \item Multi-objective reward system with 2+ metrics per objective
    \item Pareto frontier: 5+ $\alpha$ values $\times$ 2 graph types (5 seeds each)
    \item 4-6 plots: Pareto curves, efficiency-fairness scatter, Gini distributions, catch-per-detective histograms
    \item 2-3 GIFs: standard vs metro graphs in action
    \item Analysis: optimal trade-offs? When do priority edges help/hurt? Policy differences?
\end{itemize}

\sectiondivider

\section*{Shared Requirements (All Tasks)}

All teams must follow these common guidelines to ensure quality and reproducibility.

\begin{tcolorbox}[modern]
\subsection*{Code Quality}
\begin{itemize}[leftmargin=1.2em, itemsep=4pt]
    \item Follow existing code structure (inherit from base classes)
    \item Update relevant folder READMEs documenting new modules
    \item Add 2-3 unit tests for core functionality
    \item Hydra configs for all experiments
\end{itemize}

\subsection*{Experiments}
\begin{itemize}[leftmargin=1.2em, itemsep=4pt]
    \item Run on both \textbf{small} (5-7 agents, 20-30 nodes) and \textbf{large} (15-20 agents, 50-80 nodes) graphs
    \item Use fixed seeds for reproducibility (report all seeds in README)
    \item Minimum 5 seeds per experimental condition
\end{itemize}

\subsection*{Documentation}
\begin{itemize}[leftmargin=1.2em, itemsep=4pt]
    \item Update root \texttt{README.md} with new ``Semester Contribution'' section:
    \begin{itemize}[itemsep=2pt]
        \item Research question \& hypothesis
        \item Implementation summary (what was added/modified)
        \item Key results (embed 2-3 key plots/GIFs)
        \item Conclusions \& limitations
        \item Future work
    \end{itemize}
    \item Keep it concise (500-800 words max)
\end{itemize}
    \end{tcolorbox}

% =========================
% GRADING RUBRIC
% =========================
\section*{Grading Rubric}

\begin{tcolorbox}[modern, title=\ttfamily\textbf{Total: 100 points}]
Each task has the same grading structure. All requirements must use Hydra configs and be reproducible.
\end{tcolorbox}

\vspace{0.5em}

\subsection*{Implementation (40 pts)}
\begin{itemize}[leftmargin=1.4em, itemsep=3pt]
    \item Core functionality working (new modules/classes follow codebase conventions) \dotfill \textbf{20 pts}
    \item Integration with existing system (configs, training loops, evaluation) \dotfill \textbf{10 pts}
    \item Code quality (documentation, tests, readable) \dotfill \textbf{10 pts}
\end{itemize}

\subsection*{Experiments (30 pts)}
\begin{itemize}[leftmargin=1.4em, itemsep=3pt]
    \item Rigorous experimental design (controls, baselines, multiple seeds) \dotfill \textbf{10 pts}
    \item Sufficient scale (small + large graphs, 5+ conditions) \dotfill \textbf{10 pts}
    \item Reproducibility (fixed seeds, configs, clear instructions) \dotfill \textbf{10 pts}
\end{itemize}

\subsection*{Results \& Analysis (20 pts)}
\begin{itemize}[leftmargin=1.4em, itemsep=3pt]
    \item Clear visualizations (4-6 plots + 2-3 GIFs with captions) \dotfill \textbf{10 pts}
    \item Insightful analysis (hypothesis testing, failure cases, limitations) \dotfill \textbf{10 pts}
\end{itemize}

\subsection*{Documentation (10 pts)}
\begin{itemize}[leftmargin=1.4em, itemsep=3pt]
    \item Semester Contribution in root README (clear, well-structured) \dotfill \textbf{6 pts}
    \item Updated folder READMEs for modified modules \dotfill \textbf{4 pts}
\end{itemize}

\subsection*{Bonus (up to +10 pts)}
\begin{itemize}[leftmargin=1.2em, itemsep=3pt]
    \item Novel extension beyond task requirements (e.g., additional architectures, metrics, theoretical analysis) \dotfill \textbf{+5 pts}
    \item Exceptional results (paper-quality plots, surprising insights, strong baselines) \dotfill \textbf{+5 pts}
\end{itemize}

\vspace{1em}

\sectiondivider

\section*{Presentation}

Brief 10-minute presentation with slides covering: (1) Research question, (2) Implementation highlights, (3) Key results (3-4 plots/GIFs), (4) Failure analysis, (5) Future work. Live demo or pre-recorded GIF required.

\vspace{0.5em}

\section*{Submission}

\begin{tcolorbox}[modern]
\begin{itemize}[leftmargin=1.2em, itemsep=4pt]
    \item \textbf{GitHub}: Push all code, configs, and updated READMEs to repository
    \item \textbf{Canvas}: Submit PDF slides and link to GitHub branch/PR
    \item \textbf{Artifacts}: Include plots/GIFs in \codepath{src/artifacts/semester\_contribution/}
    \item \textbf{Team}: Document individual contributions in README
\end{itemize}
\end{tcolorbox}

\vspace{2em}

\noindent\rule{\textwidth}{0.5pt}

\vspace{0.5em}
\noindent\textbf{Prepared by:} Tamás Takács \\
\textbf{Semester:} 2025/26/2

\vspace{1em}
\begin{center}
    \footnotesize\color{accentcolor} Licensed under \href{https://creativecommons.org/licenses/by-nc-nd/4.0/}{CC BY-NC-ND 4.0}. \copyright~Tamás Takács, 2025.
\end{center}
 
\end{document}
